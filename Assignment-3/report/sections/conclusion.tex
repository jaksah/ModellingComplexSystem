The simulations show that the group eventually moves in the correct direction even if only a small proportion of the group is informed. 
In fact, the larger the group, the smaller the proportion of informed individuals is required in order to lead the group in the correct direction as shown in Figure \ref{fig:accuracy}.
\\\\
Furthermore, it can be observed that the portion of informed individuals has to overcome a kind of threshold before the elongation goes higher than about one. 
When the proportion of informed individuals is low, the elongation is one, but above the threshold there are enough informed individuals to guide the whole group to towards the goal. 
The elongation is at first high, due to the fact that we have the informed individuals in front and the others following them. 
With a larger portion of informed individuals, the guiding individuals increases in number and forms a wider group, bringing the elongation back towards one.
\\\\
We can observe in Figure \ref{fig:elong10}-\ref{fig:elong100} that the simulation to converges to the analytical model $\frac{1}{p\omega}$ for larger $N$.
\\\\
Our results follow the general results of \cite{theArticle} but because of the significant amount of time required for one simulation we did not manage to achieve the same resolution as Couzin et al. For the same reason we decided to not reproduce all the results in \cite{theArticle} and solely focus upon how the group accuracy and elongation depends on the proportion of informed individuals $p$ and number of individuals $N$. For future work it would be interesting to investigate in which direction the group would go if there were more than one preferred direction among the informed individuals. Couzin et al. implements a population with two preferred directions with feedback where the weight $\omega$ is increased or decreased depending on if informed members are going in the same direction or different directions, respectively. This enables the individuals to make consensus decisions. By also adding noise to one of the groups' preferred direction, \cite{theArticle} investigates what happens for different qualities of information.
\\\\
We tried to have a different approach than Couzin where we would make use of only angles instead of x- and y-directions. 
This ended up being comfortable but not really easy to use similar weights as Couzin. 
We gave it a try, but the results were quite different from Couzin, and in order to be able to make a good comparison to Couzin, we chose to go back to using x- and y-directions.
\\\\
Benjamin Nabet et al. takes a more theoretical approach to the area of decision making in animal groups. 
In \cite{anArticle} the equation for the equilibrium state is derived and plotted in a bifurcation diagram and corresponds well to the phase transition diagram in \cite{theArticle}.
\\\\
An interesting thought came to our mind as we were modeling these animal groups. I would be interesting to simulate groups of animals where some kind of raking system is present, for example if some individuals are superior to others in the group. It would then be interesting to see how many low ranked, informed, individuals it would take to move the the group towards the food and compare that to how many high ranked individuals it would take to achieve the same result, given a size of the group.