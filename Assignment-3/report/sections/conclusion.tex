The simulations show that the group eventually moves in the correct direction even if only a small proportion of the group is informed. 
In fact, the larger the group, the smaller the proportion of informed individuals is required in order to lead the group in the correct direction as shown in Figure \ref{fig:accuracy}.
\\\\
Furthermore, it can be observed that the portion of individuals to be informed has to overcome a kind of threshold before the elongation goes higher than about one. 
When the portion of informed individuals are low, the elongation is one, but after this threshold there are enough informed individuals to direct the whole group to towards the goal. 
The elongation is at first high, due to the fact that we have the informed individuals in front and the others following them. 
With a larger portion of informed individuals, the driving individuals gets larger, and thereby there will be a wider group, bringing the elongation back towards one.
\\\\
We can observe in Figure \ref{fig:elong10}-\ref{fig:elong100} that the simulation looks like converging to the analytical model, $\frac{1}{p\omega}$.
\\\\
Our results follow the general results of \cite{theArticle} but because of the significant amount of time required for one simulation we did not manage to achieve the same resolution as Couzin et al.