% A group contains $N$ individuals, where each individual $i$ has a position vector $\boldsymbol{c}_i(t)$, direction as a scalar $d_i(t)$ defined as counter-clockwise angle from the positive x-axis, and speed $s$. 
% The individuals avoid collisions with their neighbors by turning away from individuals $j$ within distance $\alpha$ according to
% \begin{equation}
% d_i (t+\Delta t) = \arg \left(- \sum_{j \neq i} \frac{\boldsymbol{c}_j (t) - \boldsymbol{c}_i (t)}{|\boldsymbol{c}_j (t) - \boldsymbol{c}_i (t) |}\right),
% \label{eq:repulsion}
% \end{equation}
%  where $\arg$ refer to angle to the positive x-axis of of the resulting vector inside the parenthesis and $d_i(t)$ is the desired direction of motion. 
%  If there are no neighbors within distance $\alpha$, the individuals will move towards individuals $j$ within distance $\rho$ according to
% \begin{equation}
% d_p (t) = \arg \left(\sum_{j \neq i} \frac{\boldsymbol{c}_j (t)- \boldsymbol{c}_i (t)}{|\boldsymbol{c}_j (t) - \boldsymbol{c}_i (t)|} \right),
% \label{eq:dp}
% \end{equation}

% \begin{equation}
% d_d (t) = \text{mean}\left( \sum_{j = 1} d_j (t) \right),
% \label{eq:dd}
% \end{equation}

% \begin{equation}
% d_i (t+\Delta t) = \frac{d_p (t) + d_d (t)}{2}.
% \label{eq:attraction}
% \end{equation}
% Here $d_p$ is the direction from individual $i$ to individual $j$, and $d_d$ is the mean value of the directions of the individuals $j$.
% A proportion $p$ of the individuals know in which direction $g$ to go, and their desired direction is given by
% \begin{equation}
% d_i^\prime (t+\Delta t) = d_i (t+\Delta t) + \omega (g - d_i (t+\Delta t)),
% \label{eq:attraction}
% \end{equation}
% where $\omega$ decides the influence of the known direction $g$; $\omega=0$ makes the individual move only by social interaction and the higher the value of $\omega$, the more the individual moves in the known direction.
% \\\\
% A small random angle $\theta$ is added to 





A group contains $N$ individuals, where each individual $i$ has a position vector $\boldsymbol{c}_i(t)$, direction vector $\boldsymbol{v}_i(t)$ and speed $s$. 
The individuals avoid collisions with their neighbors by turning away from individuals $j$ within distance $\alpha$ according to
\begin{equation}
\boldsymbol{d}_i (t+\Delta t) = - \sum_{j \neq i} \frac{\boldsymbol{c}_j (t) - \boldsymbol{c}_i (t)}{|\boldsymbol{c}_j (t) - \boldsymbol{c}_i (t) |},
\label{eq:repulsion}
\end{equation}
where $\boldsymbol{d}_i(t)$ is the desired direction of motion. 
If there are no neighbors within distance $\alpha$, the individuals will move towards individuals $j$ within distance $\rho$ according to
\begin{equation}
\boldsymbol{d}_i (t+\Delta t) = \sum_{j \neq i} \frac{\boldsymbol{c}_j (t)- \boldsymbol{c}_i (t)}{|\boldsymbol{c}_j (t) - \boldsymbol{c}_i (t)|} + \sum_{j = 1} \frac{\boldsymbol{v}_j (t)}{|\boldsymbol{v}_j (t)|}.
\label{eq:attraction}
\end{equation}
The direction $\boldsymbol{d}_i(t)$ is normalized to a unit vector $\hat{\boldsymbol{d}}_i(t)$. 
A proportion $p$ of the individuals know in which direction $\hat{\boldsymbol{g}}$ to go, and their desired direction is given by
\begin{equation}
\boldsymbol{d}_i^\prime (t+\Delta t) = \frac{\hat{\boldsymbol{d}}_i (t+\Delta t) + \omega \hat{\boldsymbol{g}}}{|\hat{\boldsymbol{d}}_i (t+\Delta t) + \omega \hat{\boldsymbol{g}}|},
\label{eq:attraction}
\end{equation}
where $\omega$ decides the influence of the known direction $\hat{\boldsymbol{g}}$; $\omega=0$ makes the individual move only by social interaction and the higher the value of $\omega$, the more the individual moves in the known direction.
\\\\
A small random angle $\theta$ is added to 