A group contains $N$ individuals, where each individual $i$ has a position vector $\boldsymbol{c}_i(t)$, direction vector $\boldsymbol{v}_i(t)$ and speed $s$. The individuals avoid collisions with their neighbors by turning away from individuals $j$ within distance $\alpha$ according to
\begin{equation}
\boldsymbol{d}_i (t+\Delta t) = - \sum_{j \neq i} \frac{\boldsymbol{c}_j (t) - \boldsymbol{c}_i (t)}{|\boldsymbol{c}_j (t) - \boldsymbol{c}_i (t) |},
\label{eq:repulsion}
\end{equation}
where $\boldsymbol{d}_i(t)$ is the desired direction of motion. If there are no neighbors within distance $\alpha$, the individuals will move towards individuals $j$ within distance $\rho$ according to
\begin{equation}
\boldsymbol{d}_i (t+\Delta t) = \sum_{j \neq i} \frac{\boldsymbol{c}_j (t)- \boldsymbol{c}_i (t)}{|\boldsymbol{c}_j (t) - \boldsymbol{c}_i (t)|} + \sum_{j = 1} \frac{\boldsymbol{v}_j (t)}{|\boldsymbol{v}_j (t)|}.
\label{eq:attraction}
\end{equation}
The direction $\boldsymbol{d}_i(t)$ is normalized to a unit vector $\hat{\boldsymbol{d}}_i(t)$. A proportion $p$ of the individuals know in which direction $\hat{\boldsymbol{g}}$ to go, and their desired direction is given by
\begin{equation}
\boldsymbol{d}_i^\prime (t+\Delta t) = (1-\omega)\hat{\boldsymbol{d}}_i (t+\Delta t) + \omega \hat{\boldsymbol{g}},
\label{eq:attraction}
\end{equation}
where $\omega$ decides the influence of the known direction $\hat{\boldsymbol{g}}$; $\omega=0$ makes the individual move only by social interaction and $\omega=1$ makes the individual move only in the known direction.